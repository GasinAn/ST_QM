\chapter{Hermitean算符和其延拓}

\section{Cayley变换方法}

\begin{theorem}
    若$\hat{A}$是Hermitean算符, 则:
    \begin{enumerate}
        \item $\hat{A}$是自伴算符, 当且仅当$R_{\hat{A}+i\hat{I}}=\H$且$R_{\hat{A}-i\hat{I}}=\H$.
        \item $\hat{A}$是本质自伴算符, 当且仅当$R_{\hat{A}+i\hat{I}}$是稠密的且$R_{\hat{A}-i\hat{I}}$是稠密的.
    \end{enumerate}
\end{theorem}
\begin{theorem}
    若$\hat{A}$是Hermitean算符, 且$\exists\,\lambda\in\R$, $R_{\hat{A}+\lambda\hat{I}}=\H$, 则$\hat{A}$是自伴算符.
\end{theorem}

\begin{definition}
    若$\hat{A}$是闭Hermitean算符, 则$\hat{W}_{\hat{A}}:=(\hat{A}-i\hat{I})(\hat{A}+i\hat{I})^{-1}$称为$\hat{A}$的Cayley变换.
\end{definition}
